% \Section{Abstract}

The rapid growth of the Internet has enabled billions of people to share photos, to blog, and to play games with each other. However, languages remain a barrier for people to connect, socialize, and cooperate around the world. 
We conducted a 12-person user study to understand how language barriers affect cooperative experiences through 3 popular cooperative games. Results showed that participants without common languages rated their experience as significantly more frustrating and less fun. We propose using body language to transcend language barriers for cooperative games. Our system, BodyTalk, uses Kinect sensors to track users' postures and presents them as avatars in real-time with other users over the Internet, and uses Wii remotes to navigate.  Our 48-person user study using our prototype game showed that adding body language to cooperative experiences was more fun and less frustrating, and improved co-experience for participants without common languages. Also, 83\% of the participants preferred having body language communication. 

%
%With the rapid development of Internet, it is common for players to play cooperative games from different country. 
%We observed that playing cooperative game with different language speaker would cause frustrating game experience.
%We suggest to use body language as a communication manner in cooperative game to get better game experience. 
%We present a game prototype, Mute Robot, which implements cooperative game design through body language communication.
%By using extended Short Feedback Questionnaire (eSFQ)\cite{eSFQ} and Cooperative Performance Metrics (CPMs)\cite{CPMs} to evaluate game experience, our user study results indicate that, using body language in cooperative game design would enhance original game experience and decrease 48\% frustrating caused by language boundary.
%After adding the communication manner of body language, 83\% and 75\% (different language group and common language group) players choose our new communication manner (``body language'' and ``both'') as the favorite manner. 
