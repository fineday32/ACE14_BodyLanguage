\section{Pilot Study}

% To understand how language boundary affect cooperative game experience, we recruit 9 Taiwaneses and 3 Japaneses and divided into 2 separate group-types(i)Taiwan-Taiwan, (ii)Taiwan-Japan. We have validated that Taiwanese players don’t understand Japanese and vise versa. We choose three different cooperative games on Steam[1] (so that player can talk to each other through Steam Talk), Rocketbirds:Hardboiled Chicken(Cooperative Action Game) ,Portal 2 (Cooperative Puzzle Game), Monaco (Cooperative Stealth Game).To emulate quick match via Internet, two players are playing these games at two distinct rooms using headphones for communication. Players play these three games 30 minutes for each. After playing each game, players will fill up an eSFQ[2] questionnaire for rapid assessment of game experiences and conduct an open-ended interviews.

In order to understand how language boundary affect cooperative game experience, we proceeded pilot study, added a factor of language boundary, for players to play the famous cooperative game on the market today. We recruited players with 9 Taiwaneses and 3 Japanese, and we divided them into 2 separated group-types: (i)Taiwan-Taiwan, (ii)Taiwan-Japan. We had confirmed that all Taiwanese players don't understand Japanese and vice versa. We choosed three different famous cooperative games: Rocketbirds:Hardboiled Chicken (Cooperative Action Game), Portal 2 (Cooperative Puzzle Game), Monaco (Cooperative Stealth Game). In addition, all three games were chosen from Steam\cite{PS1} so that players can talk to each other through Steam Talk. Besides, in order to emulate quick connection via Internet, both players were placed at two different rooms and used headphones to communicate with each other. Each time players proceed these three games for 30 minutes. Last but not least, we would like to assess game experiences and conducted an open-ended interviews, players had to fill up an eSFQ\cite{PS2} questionnaire after playing each game respectively.




