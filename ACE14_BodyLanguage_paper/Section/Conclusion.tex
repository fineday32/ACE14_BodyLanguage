\section{Conclusions}

% 我們Propose a Game Design for Body language.在此遊戲設計下,透過使用者實驗的資料分析,證明了加入肢體語言會使得遊戲的體驗上升,強烈的互動感會使玩家更沈浸于遊戲,並加強夥伴間合作的感覺,而且Body language can normalize game difficulty.使不論語言通或不通的玩家都能有一致的遊戲體驗,並提供語言不通的玩家彼此能交流的一種溝通方式,由於肢體語言是十分自然的交流方式,因此十分適合在遊戲裡使用。We hope to inspire more exploration of using body language in game designs,並將遊戲的樂趣帶給更多人。

% We propose a game design pattern for body language. Under this game design, 

% We are the first work to define the problem that cooperative game would be frustrating while playing with different language speaker. We suggest a probable solution to use body language as a communication manner. Thus we present a game prototype, Mute Robot, using body language in cooperative game design. Our user study shows that with body language consistency. Player will get more similar game experience, and body language also enhances fun and enjoyment and decrease 48\% frustrating caused by language boundary. According to our final questionnaire, 

Our 12-person study with three popular online co-operative games showed that language barrier significantly degraded players' experience. 
We proposed and developed a platform, called BodyTalk, to explore how body language communication affects only co-operative experiences. It used Kinect sensors to track users’ postures and shared them as avatars in real-time over the Internet, and uses Wii remotes to navigate the avatars. 

Our 48-person user study using our prototype game built on BodyTalk showed that adding body language to cooperative experience made it more fun and less frustrating, and improved the co-experience for all participants, especially for those without common languages. Also, 79\% of the participants preferred having body language communication.

In addition to exploring improved sensors to better support non-verbal communication, we also plan to explore how body language affects other types of co-operative experiences.


%We are the first work to define the problem that cooperative game would be frustrating while playing with different language speaker. We suggest a probable solution to use body language as a communication manner. Thus we present a game prototype, Mute Robot, using body language in cooperative game design. With body language consistency, our user study shows that players will get more similar game experience. Moreover, body language enhances fun, improves their positive co-experience by 33\% and decrease negative co-experience by 73\%. According to our final questionnaire, 
% only 17\% and 25\% (different language group and common language group) player choose traditional communication manner (speaking) as the favorite manner. 
%after adding the communication manner of body language, 83\% and 75\% (players with and without common languages) players choose our new communication manner (``Body language'' and ``Both'') as the favorite manner. 
%And we also report some interesting communication patterns. Game developer can go for a better game experience with these information.


% that players can express their emotion naturally.  
%We believe if we can make players' body language map to avatar prefectly, it . 
%because some players report that playing with body language only is too silent, so 

%On the other hand, by using body language communication manner, we want to explore whether if passing players' sound of feeling expression, such as laughing, we can enhance the game experience or not. Last but not least, we will add more different levels to increase game's diversity. And testing in multi-player environment, whether it will increase once more players' fun and enjoyment or not. In summary, not only do we want to explore a better game play experience, but we will also hope to be able to inspire more exploration of using body language in game designs and spread game entertainment for more people.


% We are the first work to define the problem that cooperative game would be frustrating while playing with different language speaker. We suggest a probable solution to use body language as a communication manner. Thus we present a game prototype, Mute Robot, using body language in cooperative game design. Our user study shows that with body language consistency. Player will get more similar game experience, and body language also enhances fun and enjoyment and decrease 48\% frustrating caused by language boundary. According to our final questionnaire, 
% after adding the communication manner of body language, 83\% and 75\% (different language group and common language groups) players choose our new communication manner (``Body language'' and ``Both'') as the favorite manner. 
% And we also report some interesting communication patterns. Game developer can go for a better game experience with these information. We hope to be able to inspire more exploration of using body language in game designs and spread game entertainment for more people.