\section{Conclusion}

% 我們Propose a Game Design for Body language.在此遊戲設計下,透過使用者實驗的資料分析,證明了加入肢體語言會使得遊戲的體驗上升,強烈的互動感會使玩家更沈浸于遊戲,並加強夥伴間合作的感覺,而且Body language can normalize game difficulty.使不論語言通或不通的玩家都能有一致的遊戲體驗,並提供語言不通的玩家彼此能交流的一種溝通方式,由於肢體語言是十分自然的交流方式,因此十分適合在遊戲裡使用。We hope to inspire more exploration of using body language in game designs,並將遊戲的樂趣帶給更多人。

% We propose a game design pattern for body language. Under this game design, 

We are the first work to define the problem that cooperative game would be frustrating while playing with differnet language speaker. We suggest a probably solution to use body language as a communication manner. Thus we present a game prototype, Mute Robot, using body language in cooperative game design. Our user study shows that with body language consistency, player will get more similar game experience, and body langeuage also enhances fun/enjoyment and decrease 48\% frustrating caused by language boundary. According to our final questionnaire, 
% only 17\% and 25\% (different language group and common language group) player choose traditional communication manner (speaking) as the favorite manner. 
after adding the communication manner of body language, 83\% and 75\% (different language group and common language group) players choose our new communication manner (``body language'' and ``both'') as the favorite manner. 
And we also report some interesting communication patterns. Game developer can make a better game experience with these information. We hope to inspire more exploration of using body language in game designs and spread game entertainment for more people.



