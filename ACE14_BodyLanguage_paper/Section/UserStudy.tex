

\section{User Study}

We conducted a user study in order to evaluate the accuracy whether our hypothesis was true or not.

......

\subsection{Method}

We divided users into two groups by whether they have the same native language or not. So there are common language and different language group respectively. Different language group are required to use their native language to communicate with each other. About our experiment, every team will have two players and be placed in two different room. Players play Mote Robot with each other through Internet connection. Totally we tested 24 groups with 48 users, inclusive of 12 common language groups and 12 different language groups.

In our experiment, we provided three different types of communication manners to play Mute Robot:
\begin{enumerate}
    \item Speaking: 
    Traditional gameplay manner, user can communicate through speaking language.
    \item Body Language: 
    players can communicate through body language.
    \item using Speaking and Body Language together: 
    players can communicate with each other through speaking language and body language.
\end{enumerate}

In order to eliminate order effects, each user played Mute Robot with counterbalance by administering the various types of game in different sequence. After playing each type of Mute Robot, players will fill out an eSFQ\cite{eSFQ} questionnaire to evaluate gameplay experience. Besides, When players finished all three types of Mute Robot, they will need to fill out another final questionnaire and proceed user interview. It needs about 1 hour to finish our user study experiment. We would conduct video recording at all experiment process and used these video to do CPMs\cite{CPMS} (see Figure~\ref{fig:US1}). , which used to evaluate user gameplay experience.

\begin{figure}[!h]
\centering
\includegraphics[width=0.9\columnwidth]{Figures/US_F1.png}
\caption{?????????}
\label{fig:US1}
\end{figure}



\subsubsection{Process}
\subsection{Observation}
\subsubsection{Communication Pattern}
We want to find out the difference between common language groups and different language group. And we divided them into three section (speaking, body language, and using speaking and body language together) for discussion.

\paragraph{speaking}
When we focus on speaking communication pattern, we compared common language groups and different language groups. We find out that language boundary happened in different langauge groups. However, although there are some language obstacles for different language groups, they still can find some ways as below to communicate with each other.

\begin{enumerate}
  \item Easier is better: users won't choose complicated or tediously long sentences. By contrast, they incline to use simple words to communicate. For example, such as ``YES'' or ``NO'', use simple words to disassemble complex instruction.

  \item Repeat Continuously: users will use the same language unceasing to express the same meaning until the other user understand.
  
  \item Emphasized tone: users transmit additional information through different tone, such as using brisk tone to express ``correct'' and using urgent tone to express ``wrong''.
  
  \item Move more: moveing avatar constantly to express some elements such as position, direction etc.
  
  \item Sound simulation: useing animal's sound to express animal.
\end{enumerate}

\paragraph{body language}
After our observation, we concluded that there is nodifference between common language groups and different language groups when they are using body language to communicate. We integrated all three types of body language communication patterns as below. 

We also observed that the Step-by-step style is most used across player groups and game stages. In spite of large diversity of body language communication, the players can still find way to communicate effectively.

\begin{enumerate}
  \item Repeat after me: player who received puzzle-solving hints would perform all the puzzle-solving actions in one go for the other player to observe and emulate. For example, in one of our game stages (Figure. 3), the 3 buttons on the floor had to be stepped on one after another in a specific order. The top player would perform the answer all at once for the underside player to repeat. 
 
  \item Step-by-Step: player who received puzzle-solving hints would command the other player to do one action at a time. The next command would not be given until the previous command was executed correctly. For instance, a player jumped in place several times in order to imply that the other player should stand on the object at the corresponding position.
                                  
  \item Pictogram: players would use their own body to express and mimic the hints. As shown in Figure 2, one participant wanted to express the letter “N” to the other player. Her solution was using her body to perform pictogram to show the character.
\end{enumerate}

\paragraph{using speaking and body language together}
In this section, we find out both groups have two communication patterns as below. The difference between gropus was the proportion of communication patterns that they used.

\begin{enumerate}
  \item Mixture usage: useing both speaking and body language together to transmit messages and assist illustration.

  \item Choosing only one pattern from speaking or body language: users will choose the most suitable or favorite communication manner, and maybe change when special condition happened.
\end{enumerate}

\begin{table}[!h]
\renewcommand\arraystretch{1.5}
  \centering
  \begin{tabular}{
  !{\vrule width2pt}p{0.22\columnwidth}
  !{\vrule width2pt}p{0.08\columnwidth}
  !{\vrule width2pt}p{0.08\columnwidth}
  !{\vrule width2pt}p{0.08\columnwidth}
  !{\vrule width2pt}p{0.08\columnwidth}
  !{\vrule width2pt}p{0.08\columnwidth}
  !{\vrule width2pt}p{0.08\columnwidth}
  !{\vrule width2pt}}
    \Xhline{2pt}
    \multicolumn{1}
    {!{\vrule width2pt}c!{\vrule width2pt}}
    {\tabhead{\multirow{2}{*}{Inter-rater}}} &
    \multicolumn{6}
    {c!{\vrule width2pt}}
    {\centering\tabhead{Kappa for Metrics}} \\
    \Xcline{2-7}{2pt}
    % \hline
    & M1 & M2 & M3 & M4 & M5 & M6 \\
    \Xhline{2pt}
    Session1 & 0.75 & 1 & 0.79 & 1 & 1 & 1 \\
    \Xhline{2pt}
    Session2 & 1 & 0.8 & 1 & 1 & 1 & 1 \\
    \Xhline{2pt}
    Session3 & 0.75 & 1 & 0.87 & 1 & 1 & 1 \\
    \Xhline{2pt}
    Session4 & 1 & 1 & 0.96 & 1 & 1 & 1 \\
    \Xhline{2pt}
    Average & 0.88 & 1.2 & 0.91 & 1 & 1 & 1 \\
    \Xhline{2pt}
  \end{tabular}
  \caption{Inter-rater Agreement (M stands for CPM)}
  \label{tab:table2}
\end{table}

\subsection{Result}
After our user study, we used eSFQ, CPMs, and Final questionnaire to evaluate the Mute Robot gameplay experience between different communication manners.

\subsubsection{eSFQ}
eSFQ\cite{eSFQ} had been proved that it is a questionnaire for rapid assessment of game experience. eSFQ capture the experienced fun/enjoyment, curiosity, and co-experience. We used five-point likert scale ranging from ``I completely agree'', ``I agree'', ``neither/nor'', ``I disagree'' to ``I completely disagree''.

\paragraph{1. fun/enjoyment}

\begin{figure}[!h]
\centering
\includegraphics[width=0.9\columnwidth]{Figures/US_Fun_Com.png}
\caption{Common Language Group}
\label{fig:US_Fun_Com}
\end{figure}

\begin{figure}[!h]
\centering
\includegraphics[width=0.9\columnwidth]{Figures/US_Fun_Dif.png}
\caption{Different Language Group}
\label{fig:US_Fun_Dif}
\end{figure}

For common language group (see Figure~\ref{fig:US_Fun_Com}), the mean of the experienced fun level for ``speaking'' was 3.58 (SD = 1.14) (5 meaning ``Yeah, fun'' - highest level of fun and 1 meaning ``Yawn, boring'' - lowest level of fun). Their game experiences were related to simple (83\% of the users), and intuitive(54\%). The mean of the experienced fun level for ``body language'' was 4.54 (SD = 0.66). Their game experiences were related to fun (79\% of the users), intuitive (54\%), exciting (46\%) and great (42\%). The mean of the experienced fun level for ``using speaking and body language together'' was 3.96 (SD = 1.00). Their game experiences were related to fun (63\% of the users), intuitive (63\%), simple (63\%) and exciting (33\%). 

For different language group (see Figure~\ref{fig:US_Fun_Dif}), the mean of the experienced fun level for ``speaking'' was 4.08 (SD = 0.97). Their game experiences were related to fun (50\% of the users), great (42\%), intuitive (38\%), simple (38\%), and confusing (33\%). The mean of the experienced fun level for ``body language'' was 4.46 (SD = 0.66). Their game experiences were related to fun (83\% of the users), intuitive (54\%), exciting (46\%), simple (38\%) and difficult (33\%). The mean of the experienced fun level for ``using speaking and body language together'' was 4.50 (SD = 0.59). Their game experiences were related to fun (75\% of the users), intuitive (67\%), exciting (42\%), simple (42\%), great (34\%) and confusing (33\%).

The fun/enjoyment rate of ``body language'' and ``using speaking and body language together'' were better than ``speaking'', no matter in common language group or in different language group. In sum, ``body language'' got the best rate of fun level, and the rate of ``using speaking and body language together'' was only slightly lower than ``body language'' but is still very good. And ``speaking'' got the worst rate. We dicovered that the difficulty of the game would make an influence on fun. In other words, both index were in direct proportion. For instance, when the ratio of simple becomes higher, the ratio of fun will be lower than before. In addition, different language group had higher experienced fun level for ``speaking'', and it also make users feeling confused for ``speaking'' and``using speaking and body language together''.



\paragraph{2. curiosity}


\begin{figure}[!h]
\centering
\includegraphics[width=0.9\columnwidth]{Figures/US_Curi_Com.png}
\caption{Common Language Group}
\label{fig:US_Curi_Com}
\end{figure}

\begin{figure}[!h]
\centering
\includegraphics[width=0.9\columnwidth]{Figures/US_Curi_Dif.png}
\caption{Different Language Group}
\label{fig:US_Curi_Dif}
\end{figure}


\paragraph{3. co-experience}

\subsubsection{CPMs}
\subsubsection{Final Questionnaire}

% For common language group:
% The mean of the experienced fun level for “speaking” was 3.58(SD = x,xx) (5 meaning “Yeah, fun” – highest level of fun and 1 meaning “Yawn, boring” – lowest level of fun). 38\% of the users indicated that they wanted to play the game again, 58\% indicated maybe and only 4\% did not want to play again.Their game experiences were related to simple (83 \% of the users), and intuitive (54 \%). 
% The mean of the experienced fun level for “body language” was 4.54(SD = x,xx). 83\% of the users indicated that they wanted to play the game again, 13\% indicated maybe and only 4\% did not want to play again.Their game experiences were related to fun (79 \% of the users), intuitive (54 \%), exciting (46\%) and great (42 \%).
% The mean of the experienced fun level for “both” was 3.96(SD = x,xx). 63\% of the users indicated that they wanted to play the game again, 29\% indicated maybe and only 8\% did not want to play again.Their game experiences were related to fun (63 \% of the users), intuitive (63 \%), simple (63 \%) and exciting (33 \%).   

% For non-common language group:
% The mean of the experienced fun level for “speaking” was 4.08 (SD = x,xx). 54\% of the users indicated that they wanted to play the game again, 38\% indicated maybe and only 8\% did not want to play again.Their game experiences were related to fun (50 \% of the users), great (42 \%), intuitive (38 \%) ,simple (38 \%) and confusing (33 \%). 
% The mean of the experienced fun level for “body language” was 4.46(SD = x,xx). 75\% of the users indicated that they wanted to play the game again, 17\% indicated maybe and only 8\% did not want to play again.Their game experiences were related to fun (83 \% of the users), intuitive (54 \%), exciting (46 \%), simple (38 \%) and difficult (33 \%). 
% The mean of the experienced fun level for “both” was 4.5(SD = x,xx). 63\% of the users indicated that they wanted to play the game again, 37\% indicated maybe and no one did not want to play again.Their game experiences were related to fun (75 \% of the users), intuitive (67 \%), exciting (42 \%), simple (42 \%), great (34 \%) and confusing (33 \%). 
% (要改 強調難度)
% No matter common language or non-common language group,the fun/enjoyment rating of ”body language” and “both” were better than ”speaking”.Overall,”body language” got the best rating , and the rating for “both” only slightly lower but still very good, “speaking” got the worst rating. 
% For non-common language group , users thought speaking with different language was interesting and made this game harder,so the rating for “speaking” was better than common language group.And users feel confusing for “speaking” and “both”,在”body language”則沒有

\subsection{Discussion}


