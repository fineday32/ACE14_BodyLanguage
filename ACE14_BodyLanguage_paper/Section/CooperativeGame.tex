\section{Cooperative Game}

% Cooperative design has been an integral part of many games. With the success of games like Left4Dead,Portal 2,Gears of War, many game designers and producers are currently exploring the addition of cooperative patterns within their games.
% Cooperative Games encourage participation and collaboration, the goal is not to win as player but as a team of players. Discovering effective cooperative game patterns is an elusive and important problem.

In a growing diversity of generations, cooperative game has been an integral part of today's game development. There are a variety of great cooperative game such as Left4Dead, Portal 2, and Gears of War. With so many success cooperative game in the market, We can find out that it is a trend for game designers and related company to explore more and more cooperative patterns within their games. Cooperative games not only make players' collaboration but also result in players' higher game participation. To design a great cooperative game, the main idea is to win the game with team members rather than a single person. As we mentioned above, we know that it  is important for game designers to find out great cooperative patterns, but it also becomes a elusive problem to discover an effective cooperative game patterns.

% Some researchers analyzed a set of cooperative games to develop cooperative game design patterns. Zagal et al., for example, explored cooperative patterns within board games[1]. Also, Bjork and Holopainen presented a large number of game design patterns [2], which included cooperative and social interaction patterns.In addition Zagal et al, presented an ontology for analyzing game play [3]. Rocha et al. presented a framework of several cooperative game design patterns[4] .

In order to develop cooperative game design patterns, several researchers analyzed a set of cooperative games. For example, Zagal et al. explored cooperative patterns within board games \cite{1}, and it also presented an ontology with a view to analyzing game play \cite{3}; Bjork and Holopainen, which presented a large quantity of game design patterns \cite{2}, included cooperative and social interaction patterns; Rocha et al. presented a framework of several cooperative game design patterns \cite{4}.

% Special cases in cooperative game also become a exploring area. Wolmet et al.reports on a study of how parents and children play several cooperative co-located games with different characteristics[5]. Mark et al [6]. evaluate the communicative and cooperative behavior of same-age and mixed-age pairs (Young-Young, Young-Old, Old-Old),and identified noticeable differences between the group-types. Hamilton et al [7], explored how to design games for children with cerebral palsy to play ,and presents several cooperative gameplay prototypes.

We can realize some principle to generate cooperative game design patterns from many general cases. However, if we are eager to comprehend cooperative game design completely, some special cases in cooperative game are also an inevitable exploring area. For instance, Wolmet et al. reported a study of how parents and children play several cooperative co-located games with different characteristics \cite{5}; Mark et al. \cite{6} evaluated the communicative and cooperative behavior of same-age and mixed-age pairs (Young-Young, Young-Old, Old-Old), and identified noticable difference between group types; Hamilton et al. \cite{7} explored how to design games for children to play with cerebral palsy, and it also presented several cooperative gamplay prototypes.

% With rapid development of Internet, playing cooperative game with player from different country become a normal situation, but cooperative game for different language speaker is still an untapped area. In this work, we want to understand how language boundary affect cooperative game experience, and how to design a game that avoid negative game experience while playing with different language speaker.

With the rapid development of Internet, it is common for players to play cooperative games from different country. Nonetheless, players consist of different language speakers are still an untapped area for cooperative game design. In our work, we want to understand how language boundary affects cooperative game experience and how to design a game which can avoid negative gameplay experience while playing with different language speakers.

