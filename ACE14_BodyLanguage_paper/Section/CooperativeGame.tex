\section{Related Work}

\subsection{Cooperative Game Design}

% In a growing of game industry, cooperative game has been an integral part of today's game development. There are a variety of great cooperative games such as Left4Dead, Portal 2, and Gears of War. With so many successful cooperative games in the market, We can find out that it is a trend for game designers and related company to explore more and more cooperative patterns within their games. Cooperative games encourage participation and collaboration, the goal is not to win as player but as a team of players. As we mentioned above, we know that it  is important for game designers to find out great cooperative patterns, and it also becomes an elusive problem to discover an effective cooperative game patterns.

% Some researchers analyzed a set of cooperative games to develop cooperative game design patterns. Zagal et al., for example, explored cooperative patterns within board games[1]. Also, Bjork and Holopainen presented a large number of game design patterns [2], which included cooperative and social interaction patterns.In addition Zagal et al, presented an ontology for analyzing game play [3]. Rocha et al. presented a framework of several cooperative game design patterns[4] .

Several prior work have explored and analyzed cooperative game design patterns. Zagal et al. explored cooperative patterns within board games and yielded some observations that game designers might consider useful for designing collaborative game\cite{CG1}, and it also presented an ontology with a view to analyzing game play. \cite{CG3}. 
Bjork and Holopainen presented a large quantity of game design patterns \cite{CG2}, including cooperative and social interaction patterns.
Rocha et al. presented a framework of several cooperative game design patterns and analyzed the actual impact of using these game mechanics to design a cooperative video game \cite{CG4}.
El-Nasr et al. extended Rocha et al.'s model and proposed Cooperative Performance Metrics (CPM) to evaluate game experience \cite{CPMs}.
% Special cases in cooperative game also become a exploring area. Wolmet et al.reports on a study of how parents and children play several cooperative co-located games with different characteristics[5]. Mark et al [6]. evaluate the communicative and cooperative behavior of same-age and mixed-age pairs (Young-Young, Young-Old, Old-Old),and identified noticeable differences between the group-types. Hamilton et al [7], explored how to design games for children with cerebral palsy to play ,and presents several cooperative gameplay prototypes.

%We can realize some principle to generate cooperative game design patterns from many general cases. However, if we are eager to comprehend cooperative game design completely, some special cases in cooperative game are also an inevitable exploring area. 
Wolmet et al. reported a study of how parents and children play several cooperative co-located games with different characteristics \cite{CG5}. 
Mark et al. \cite{CG6} evaluated the communicative and cooperative behavior of same-age and mixed-age pairs (Young-Young, Young-Old, Old-Old), and identified noticable difference between group types. 
Hamilton et al. \cite{CG7} explored how to design games for children to play with cerebral palsy, and it also presented several cooperative gameplay prototypes.

% With rapid development of Internet, playing cooperative game with player from different country become a normal situation, but cooperative game for different language speaker is still an untapped area. In this work, we want to understand how language boundary affect cooperative game experience, and how to design a game that avoid negative game experience while playing with different language speaker.

% With the rapid development of Internet, it is common for players to play cooperative games from different country. Nonetheless, players consist of different language speakers are still an untapped area for cooperative game design. In our work, we want to understand how language boundary affects cooperative game experience and how to design a game which can avoid negative gameplay experience while playing with different language speakers.

In our work, we explored and evaluated the possibility to use body language as a communication manner in cooperative game design, and analyzed the communication pattern with players.

\subsection{Body Language}

% Human communication involves not only speech, but also a wide variety of gestures and body motions. Body language is the most effective means of communicating un-spoken emotions, a non-verbal way to impart your thoughts without verbalizing it. According to The 7\% Rule [1], communication is only 7 percent verbal and 93 percent non-verbal. The non-verbal component was made up of body language(55 percent) and tone of voice (38 percent).

Consist of human communication, there is not only speech but also inclusive of various gestures and body motions. Body language, a non-verbal way to transmit your thoughts without verbalizing. According to The 7\% Rule\cite{GD2}, the influence of communication for verbal is only 7\% but is 93\% for non-verbal expression. And the non-varbal expression is made up of body language (55\%) and tones of voice (38\%).

% Charades[2] is a word guessing game.In the form most played today, it is an acting game in which one player acts out a word or phrase, often by miming similar sounding words, and the other players guess the word or phrase. The idea is to use physical rather than verbal language to convey the meaning to another party.

Charades\cite{GD3} is a word guessing game. It is an acting game in which one player act as a word or a phrase, and sometimes imitates a similar pronounced words, while the other players guess the answer. The main idea is to use the body to make physical expression rather than using verbal language. 

% Inspired by The 7\% Rule and Charades, We suggest to use body language as a communication manner in cooperative game to normalize player’s communication skill, so that no matter players are playing with different language speaker or not, their communication skill is near enough for game developer to design a proper difficulty to please players.

Inspired by The 7\% Rule and the Charades, we suggested using body language as a communication manner in cooperative game to normalize player's communication skill. With this idea, whether players are playing with different language speakers or not, their communication skill is near enough for a game developer to design a proper difficulty to entertain players. On the other hand, many researchers have argued that the body movement brings about a positive emotional and social response \cite{GD7, GD8, GD9}. We believe that body language communication should enhance game engagement and enjoyment.
